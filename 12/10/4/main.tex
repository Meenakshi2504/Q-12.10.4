\let\negmedspace\undefined
\let\negthickspace\undefined
\documentclass[journal,12pt,onecolumn]{IEEEtran}
\usepackage{cite}
\usepackage{amsmath,amssymb,amsfonts,amsthm}
\usepackage{algorithmic}
\usepackage{graphicx}
\usepackage{textcomp}
\usepackage{xcolor}
\usepackage{txfonts}
\usepackage{listings}
\usepackage{enumitem}
\usepackage{mathtools}
\usepackage{gensymb}
\usepackage{comment}
\usepackage[breaklinks=true]{hyperref}
\usepackage{tkz-euclide} 
\usepackage{listings}
\usepackage{gvv}                                        
\def\inputGnumericTable{}                                 
\usepackage[latin1]{inputenc}                                
\usepackage{color}                                            
\usepackage{array}                                            
\usepackage{longtable}                                       
\usepackage{calc}                                             
\usepackage{multirow}                                         
\usepackage{hhline}                                           
\usepackage{ifthen}                                           
\usepackage{lscape}

\newtheorem{theorem}{Theorem}[section]
\newtheorem{problem}{Problem}
\newtheorem{proposition}{Proposition}[section]
\newtheorem{lemma}{Lemma}[section]
\newtheorem{corollary}[theorem]{Corollary}
\newtheorem{example}{Example}[section]
\newtheorem{definition}[problem]{Definition}
\newcommand{\BEQA}{\begin{eqnarray}}
\newcommand{\EEQA}{\end{eqnarray}}
\newcommand{\define}{\stackrel{\triangle}{=}}
\theoremstyle{remark}
\newtheorem{rem}{Remark}
\begin{document}

\bibliographystyle{IEEEtran}
\vspace{3cm}

\title{12.10.4}
\author{EE22BTECH11008 - Annapureddy Siva Meenakshi$^{*}$% <-this % stops a space
}
\maketitle
\bigskip

\renewcommand{\thefigure}{\theenumi}
\renewcommand{\thetable}{\theenumi}
Q:In a Young's double-slit experiment, the slits ar e separated by
$0.28 mm$ and the screen is placed 1.4 m away. The distance between
the central bright fringe and the fourth bright fringe is measured
to be $1.2 cm$. Determine the wavelength of light used in the
experiment.
\\\solution\\
\begin{table}[!ht]
    \centering
        \begin{tabular}{|c|c|c|}
    \hline
      \textbf{Variable}& \textbf{Description}& \textbf{Value}\\\hline
    d& Distance between two slits& $28 \times 10^{-5} m$ \\\hline
    $\lambda$& wavelength of light & none \\\hline
    m & order of fringe&4\\\hline
   $ \theta $& Angle between central maxima and $n_{th}$ fringe & none\\\hline
    $\Delta x $& Path difference between waves & none\\\hline
    L & Distance between screen and slits & $1.4m$\\\hline
    $\Delta y_m $& Distance between central maxima and $m_{th}$ fringe & none\\ 
    \hline
    $\Delta y_4 $& Distance between central maxima and $4_{th}$ fringe &$12\times 10^{-3}m$\\ 
    \hline
  \end{tabular}

    \caption{input parameters}
    \label{tab:12_10_4_1}
\end{table}
\text{The distance between the central bright fringe and the $m$-th bright fringe is given by the formula:}
\begin{align}
\Delta y_m &= m \frac{\lambda L}{d} \\
\lambda&=\frac {\Delta y_m d}{mL}
\end{align}

\begin{align}
  \therefore\ \lambda&=\frac {\Delta y_4 d}{mL}\\
&=\frac{12\times 10^{-3}\times28 \times 10^{-5}}{4\times1.4}\\
&=6\times 10^{-7}\\
&=600nm
\end{align}
Therefore,the value of wavelength is $600nm$.
\end{document}
